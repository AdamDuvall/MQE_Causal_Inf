\documentclass[a4paper, 10pt]{article}
\usepackage[left=2.25cm,right=2.25cm,top=2.25cm,bottom=2.25cm]{geometry}
\usepackage{mathpazo}
\usepackage{amsfonts, amssymb, amsmath}
\usepackage{fancyhdr}
\usepackage{enumerate}
\usepackage{graphicx}
\usepackage{xfrac}
\usepackage{lscape}
\usepackage[dvipsnames]{xcolor}
\usepackage{setspace}
\usepackage{subfig}
\usepackage{booktabs}
\usepackage{blindtext}
\usepackage{epsfig}
\usepackage{hyperref}
\usepackage{booktabs,caption,fixltx2e}
\usepackage[flushleft]{threeparttable}
\usepackage{outlines}
\newcommand{\beqn}{\begin{eqnarray*}}
	\newcommand{\eeqn}{\end{eqnarray*}}
\newcommand{\bi}{\begin{itemize}}
	\newcommand{\ei}{\end{itemize}}
\newcommand{\bnot}{\hat{\beta}_0}
\newcommand{\bone}{\hat{\beta}_1}

\newcommand{\claire}[1]{{\color{red} [CD: {#1}]}}

\geometry{ left = 1in, right = 1in, top =1in, bottom =1in }
\pagestyle{fancy}
\setlength{\headsep}{.4 in}
\setlength{\parskip}{0.2 cm}
\setlength{\parindent}{1cm}
\renewcommand{\headrulewidth}{.25pt}
\lhead{MQE: Economic Inference from Data \\Fall 2020}
\chead{}
\rhead{Claire Duquennois}
\lfoot{}
\cfoot{}
\rfoot{\footnotesize \thepage}
\newenvironment{myindentpar}[1]%
 {\begin{list}{}%
         {\setlength{\leftmargin}{#1}}%
         \item[]%
 }
 {\end{list}}

\begin{document}


\vspace{10mm}
\linespread{1.25}
\bigskip
Dear MQE class of 2020,\\ 


This country, and the wider world, is grappling with deep seated racial issues and long overdue questioning and self-examination on how racism is deeply embedded in our lives and behaviors. Like many institutions throughout the country, academic institutions are also taking part in this reflection. 

In the Economics field, these events have given rise to conversations and self-reflection. Economics struggles with a diversity problem. White men continue to dominate the field. Black and Brown economists, and to a lesser extent women, continue to be underrepresented. Those that do enter the field face the constant battles, some large, some small, that come hand in hand with being a minority in one's place of work. This is particularly concerning given that economics is social science. How are we to provide scientific insights into our social systems and policies if we have only experienced society from a position of racial (and/or gender) privilege?

I love this field, and as your professor, I hope I can communicate and transmit my passion to you. I believe that the econometric tool set that you are going to learn in this course is very powerful. Thus far, these powerful tools have often been yielded by privileged people with little insight into the lived truths of minorities. This has shaped the types of questions economists ask, the way we think about questions and the assumptions we make when answering them. Importantly, I fear that biases in our discipline go on to contaminate the industries and institution to which we send our graduates. 

Thus, in my continued effort to improve myself and to push for improvement in this field, I offer an apology, some promises, and a few requests.  


I apologize for mistakes I know I am going to make. For years I have been training to think and analyze as an economist. This training has been incredibly valuable, but I also know that it may lead me to fail at questioning certain problematic assumption that have been accepted as normal in this field.  I need your fresh eyes to help me see these. If a statement, question, or assumption strikes you as odd and questionable or not reflective of your experience, let's discuss it and work towards change rather than accepting silently. Furthermore, if I ever say or do something that you feel is inappropriate, please let me know. Give me the opportunity to apologize and learn from your experience so I can avoid making the same mistakes in the future. Know that it is coming from ignorance, not intent, and though the duty to speak up places a burden on you, future students will benefit from it.

I promise to listen. I will be setting some time aside every week for student "coffee hour", to get to know you and discuss anything you would like, other than course materials and problem sets which can be addressed during my usual office hours. You are welcome to schedule some private time or come in groups. If you are interested in discussing diversity issues and resources in economics, I would be happy to do so.  

Finally, I request that you listen, and listen equally. Too often the voices of under-represented minorities and women get drowned out. This can happen because of publication bias that fails to value research on minorities or the research of minority faculty. It can happen when researchers don't cite the work of minorities.  It can happen because when you are the only person of color in an office, or class, you have no one to talk to about your experiences. It can happen when otherwise well meaning individuals interrupt certain groups more frequently, or repeat a contribution or idea originally voiced by someone without crediting them. If I do this accidentally, let me know. If someone points out that you engaged in any of these behaviors, exercise grace: listen, apologize, and learn from it.  If a classmate stated an important point that got ignored, amplify their voice by restating it while crediting them. These are behaviors that are hard to change, but with awareness and a concerted effort we can try. 
\vspace{3mm}

\noindent Sincerely, 

\noindent Claire Duquennois











\end{document}

